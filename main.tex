%\documentclass[twocolumn]{sig-alternate}
\documentclass{sig-alternate-10pt}
%\documentclass[twocolumn,10pt]{infocom}
%\documentclass[twocolumn,10pt]{IEEEtran_v15}
%\documentclass[twocolumn,10pt]{IEEEtran}
%\documentclass[twocolumn,10pt]{article}
%\usepackage[sort,nocompress,space]{cite}
%\documentclass[twocolumn]{sig-alt-hotnetscr}
\usepackage{url}
\usepackage[sort,space]{cite}
\usepackage{lineno}
\renewcommand\linenumberfont{\normalfont\bfseries\small}
\usepackage{pifont}
\usepackage{epsfig,epsf,url,amssymb}
\usepackage{tabularx}
\usepackage[boxed]{algorithm2e}
\usepackage{amsmath}
\usepackage{mathtools}
\newtheorem{mydef}{Definition}
\usepackage{rotating}
%\usepackage{color}
%\usepackage{amsthm}
%\usepackage{amssymb}
%\usepackage[numbers,sort&compress]{natbib}
\usepackage{wrapfig}
\usepackage{times}
\long\def\comment#1{}
\usepackage{multirow}
\usepackage{verbatim}
\usepackage{lscape}
\usepackage{stmaryrd}
\usepackage{hhline}
\usepackage{textcomp,booktabs}
\usepackage[usenames,dvipsnames]{color}
\usepackage{colortbl}
\definecolor{mygray}{gray}{.9}
\definecolor{mypink}{gray}{.9}
\definecolor{mycyan}{cmyk}{.3,0,0,0}
\usepackage{epstopdf}
\usepackage[bf,skip=0pt]{caption}

\renewcommand{\captionfont}{\bf}
%\usepackage{subfigure}

\newcommand{\bbR}{\mathbb{R}}
\newcommand{\calN}{\mathcal{N}}
\newcommand{\calR}{\mathcal{R}}
\newcommand{\calV}{\mathcal{V}}
\newcommand{\eg}{{\it e.g.}}
\newcommand{\etal}{{\it et al.~}}
\newcommand{\etc}{{\it etc.}}
\newcommand{\ie}{{\it i.e.}}
\newcommand{\tablecapspace}{{\vspace{-0.1in}}}
\newcommand{\tablespace}{{\vspace{-0.05in}}}
\newcommand{\picspace}{{\vspace{-0.1in}}}
\renewcommand{\baselinestretch}{1}
\renewcommand{\arraystretch}{1.05}      % make the space between tabular lines larger
\newcommand{\capspace}{}           % control space between figure/table and caption
\newcommand{\paragraphb}[1]{\vspace{0.05in}\noindent{\bf #1}}

\newcommand{\paraspace}{\vspace{0.05in}}
\newcommand{\parab}[1]{\paraspace\noindent{\bf #1} }
\newcommand{\parae}[1]{\paraspace\noindent{\em #1} }
\newcommand{\parabe}[1]{\paraspace\noindent{\bf \em #1} }

%\newcommand{\parab}[1]{\noindent{\bf #1} }
%\newcommand{\parae}[1]{\noindent{\em #1} }
%\newcommand{\parabe}[1]{\noindent{\bf \em #1} }

\def\TODO#1{\textcolor{red}{TODO: #1}}

\setlength{\textheight}{9.3in}
\setlength{\columnsep}{1.4pc}
\setlength{\textwidth}{7.1in}
%\linespread{.98}

\newcommand{\sys}{{\textsf{Karuna}}\xspace}

\newcommand{\subcaption}[1]{\centerline{#1}\vspace{0.1in}}
\long\def\comment#1{}
\newtheorem{theorem}{Theorem}
\newtheorem{lemma}[theorem]{Lemma}
\newtheorem{proposition}[theorem]{Proposition}
\newtheorem{corollary}[theorem]{Corollary}

\newenvironment{icompact}{
  \begin{list}{$\bullet$}{
    \parsep 1pt plus 1pt
    \partopsep 1pt plus 1pt
    \topsep 1pt plus 2pt minus 1pt
    \itemsep 1.5pt plus 1pt
    \parskip 0pt plus 2pt
    \leftmargin 0.15in}
       }
  {\normalsize\end{list}}

\setlength{\parskip}{0cm}
\setlength{\parindent}{1em}

%\setlength{\oddsidemargin}{-0.25in}
%\setlength{\oddsidemargin}{-0.3in}
%\addtolength{\oddsidemargin}{-0.1in}
%\oddsidemargin=-0.1in % leftmargin is 1 inch

%%%%%%%% to calculate the time %%%%%%%%%%%%%%%%%%%%%%%%%%%%
\newcount\hour \newcount\minute
\hour=\time  \divide \hour by 60
\minute=\time
\loop \ifnum \minute > 59 \advance \minute by -60 \repeat
\def\drafttime{\ifnum \hour<13 \number\hour:%
                      \ifnum \minute<10 0\fi
                      \number\minute
                      \ifnum \hour<12 \ AM\else \ PM\fi
         \else \advance \hour by -12 \number\hour:%
                      \ifnum \minute<10 0\fi
                      \number\minute \ PM\fi}
\def\timestamp{\today \ \drafttime}

\usepackage{multirow}
\usepackage{rotating}

\def\sharedaffiliation{%
\end{tabular}
\begin{tabular}{c}}

\begin{document}

%\conferenceinfo{SIGCOMM,} {XXXXXX, XXXXX, XXXXX, XXXXX.} %
%\CopyrightYear{2009} \crdata{1-59593-308-5/06/0009}

\title{Title}
\author{Paper \#\#, \#\# pages, \timestamp}
%\author{\timestamp}
%\numberofauthors{5} %  in this sample file, there are a *total*
%% of EIGHT authors. SIX appear on the 'first-page' (for formatting
%% reasons) and the remaining two appear in the \additionalauthors section.
%%
%\author{
%% You can go ahead and credit any number of authors here,
%% e.g. one 'row of three' or two rows (consisting of one row of three
%% and a second row of one, two or three).
%%
%% The command \alignauthor (no curly braces needed) should
%% precede each author name, affiliation/snail-mail address and
%% e-mail address. Additionally, tag each line of
%% affiliation/address with \affaddr, and tag the
%% e-mail address with \email.
%%
%% 1st. author
%\alignauthor
%Li Chen\\
%       \affaddr{SING Lab, HKUST}
%       \email{lchenad@ust.hk}
%% 2nd. author
%\alignauthor
%Kai Chen\\
%       \affaddr{SING Lab, HKUST}
%       \email{kaichen@cse.ust.hk}
%% 3rd. author
%\alignauthor Wei Bai\\
%       \affaddr{SING Lab, HKUST}
%       \email{wbaiab@ust.hk}
%\and  % use '\and' if you need 'another row' of author names
%% 4th. author
%\alignauthor Haitao Wu\\
%       \affaddr{Microsoft}
%       \email{hwu@microsoft.com}
%% 5th. author
%\alignauthor Mohammad Alizadeh\\
%       \affaddr{Cisco}
%       \email{moattar@cisco.com}
%}

\maketitle
\setlength{\textfloatsep}{1pt}

%\vspace{-0.05in}
%\input{sec/abstract.tex}
\input{sec/introduction.tex}
%\input{sec/motivation.tex}
%\input{sec/design.tex}
%\input{sec/optimization.tex}
%\input{sec/implementation.tex}
%\input{sec/evaluation.tex}
%\input{sec/discussion.tex}
%\input{sec/relatedwork.tex}
%\input{sec/conclusion.tex}
%\input{sec/appendix.tex}
%\begin{scriptsize}
%\vspace{-0.05in}
%\begin{small}
\clearpage
\bibliographystyle{acm}
\bibliography{reference}
%\end{small}
%\clearpage
%\end{scriptsize}
%\clearpage
%\input{sections/appendix.tex}

\end{document}
